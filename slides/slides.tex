\documentclass[t,aspectratio=169,usenames,dvipsnames,xcolor=table]{beamer}
\usepackage{redhat-beamer/redhat}
\usepackage{layout}
\usepackage{xcolor}
\usepackage{ulem}
\usepackage{pifont}
\usepackage{makecell}


\definecolor{hlcolor}{RGB}{2,65,77}
\definecolor{bgcolor}{RGB}{241,241,241}
\definecolor{rhreddark}{RGB}{163,0,0}

\newcommand{\hll}[1]{{\color{rhrednew} #1}}
\newcommand{\hl}[1]{\textbf{\hll{#1}}}

%
% font
%
\usepackage{fontspec}
\usepackage{fontawesome}
\setmonofont{FreeMono}

\usepackage{tikz}
\usetikzlibrary{positioning,calc}

\usepackage{listings}
\usepackage{lstautogobble}
\lstset{
    language=C,
    basicstyle={\tiny\ttfamily},
    autogobble=true,
    moredelim=**[is][\only<1>{\color{rhrednew}\bfseries}]{^}{^},
    moredelim=**[is][\only<2->{\color{rhrednew}\bfseries}]{@}{@},
    moredelim=**[is][\only<2->{\color{blue}\bfseries}]{~}{~},
}

\renewcommand<>{\sout}[1]{
  \alt#2{\beameroriginal{\sout}{#1}}{#1}
}

\tikzset{onslide/.code args={<#1>#2}{%
  \only<#1>{\pgfkeysalso{#2}} % \pgfkeysalso doesn't change the path
}}

\title{Advanced Git}
\subtitle{IVS demonstration exercise}
\author{Viktor Malík, Petr Stodůlka, Pavel Odvody}
\institute{Red Hat}
\date{April 14, 2020}

\begin{document}

\maketitle

\begin{frame}
  \frametitle{Prerequisites}
  \begin{itemize}
    \setlength\itemsep{.3em}
    \item Basic knowledge of Git commands for:
      \begin{itemize}
      \setlength\itemsep{.3em}
        \item creating commits (\texttt{git add, git commit})
        \item inspecting current state (\texttt{git status, git diff})
        \item inspecting history (\texttt{git log, git show})
        \item working with remotes (\texttt{git pull, git push})
        \item working with branches (\texttt{git checkout, git branch})
      \end{itemize}
  \end{itemize}
\end{frame}

\begin{frame}
  \frametitle{Let's start}
  \begin{itemize}
    \item We'll write a simple tool for counting characters, words, and lines
      in a file (similar to the \texttt{wc} utility)
    \item We start with a pre-initialized repo containing very basics of the
      tool:\\
      \url{https://github.com/viktormalik/git-workshop}
    \item The repo contains a source file \texttt{wc.c}, a testing file, and 
      a \texttt{Makefile}
    \item We start by adding \texttt{.gitignore} and commiting it
  \end{itemize}
\end{frame}

\begin{frame}
  \frametitle{Current status of the repo}
  \centering
  \includegraphics[scale=0.8]{../git-figures/01-initial.pdf}
\end{frame}

\begin{frame}
  \frametitle{Basic team synchronisation}
  \vspace{-1em}
  Every member implements a different feature in their \textit{master}
  \vspace{1em}
  \begin{center}
    \includegraphics[scale=0.6]{../git-figures/02-lines.pdf}
    \hspace{6em}
    \includegraphics[scale=0.6]{../git-figures/02-words.pdf}
  \end{center}
\end{frame}

\begin{frame}
  \frametitle{Basic team synchronisation}
  \vspace{-1em}
  The second one to push must do a merge (and resolve a merge conflict)
  \vspace{1em}
  \begin{center}
    \includegraphics[scale=0.6]{../git-figures/02-after-merge.pdf}
  \end{center}
\end{frame}

\begin{frame}
  \frametitle{Better team synchronisation}
  \begin{itemize}
    \item \textbf{This is not a good practice!}
    \item Always implement new features in \textbf{separate branches}.
    \item Potential merge conflicts should be resolved in the feature branch.
    \item Ideally, merging into master should be always done using \textbf{pull
      requests}
      \begin{itemize}
        \setlength\itemsep{.2em}
        \item They allow other team members to comment on the changes
        \item Changes can be \textbf{reviewed} before they get into master
        \item Master always contains a working and approved version of the
          project
      \end{itemize}
  \end{itemize}
\end{frame}

\begin{frame}
  \frametitle{Using a feature branch}
  \vspace{-1em}
  Let us add help into the tool using a separate branch \textit{add\_help}
  \vspace{1em}
  \begin{center}
    \includegraphics[scale=0.6]{../git-figures/03-help.pdf}
  \end{center}
\end{frame}

\begin{frame}
  \frametitle{Using a feature branch}
  \vspace{-1em}
  The state of \textit{master} after \textbf{rebase}:
  \vspace{1em}
  \begin{center}
    \includegraphics[scale=0.6]{../git-figures/03-after-rebase.pdf}
  \end{center}
\end{frame}

\begin{frame}
  \frametitle{Moving branches}
  \vspace{-1em}
  We have 2 branches pointing to the same commit and we want to move one
  backwards.
  \vspace{1em}
  \begin{center}
    \hspace{-2em}
    \includegraphics[scale=0.7]{../git-figures/04-start.pdf}
  \end{center}
\end{frame}

\begin{frame}
  \frametitle{Moving branches}
  \vspace{-1em}
  This can be done using\, \texttt{git reset HEAD \^}
  \vspace{1em}
  \begin{center}
    \hspace{-2em}
    \includegraphics[scale=0.7]{../git-figures/04-after-reset.pdf}
  \end{center}
\end{frame}

\begin{frame}
  \frametitle{Moving branches}
  \vspace{-1em}
  After adding a new commit to \textit{options-opt}:
  \vspace{1em}
  \begin{center}
    \hspace{-2em}
    \includegraphics[scale=0.6]{../git-figures/04-separate-branches.pdf}
  \end{center}
\end{frame}

\begin{frame}
  \frametitle{Moving branches}
  \vspace{-1em}
  \textit{options-opt} can be now merged into master while
    \textit{own-separator} remains a feature branch in development.
  \vspace{1em}
  \begin{center}
    \hspace{-2em}
    \includegraphics[scale=0.6]{../git-figures/04-after-merge.pdf}
  \end{center}
\end{frame}

\begin{frame}
  \frametitle{Rebasing feature branches}
  \vspace{-1em}
  We add more commits to the feature branch and then \textbf{rebase} it onto
  \textit{master} (to avoid creation of a merge commit).
  \vspace{1em}
  \begin{center}
    \hspace{-2em}
    \includegraphics[scale=0.6]{../git-figures/04-after-rebase.pdf}
  \end{center}
\end{frame}

\begin{frame}
  \frametitle{Rebasing feature branches}
  \vspace{-1em}
  We made a mistake during rebase, which we had to fix with an additional
  commit.
  \vspace{1em}
  \begin{center}
    \hspace{-2em}
    \includegraphics[scale=0.6]{../git-figures/04-after-fix.pdf}
  \end{center}
\end{frame}

\begin{frame}
  \frametitle{Rebasing feature branches}
  \vspace{-1em}
  It is possible to merge the ``fix commit'' into one of the previous commits
  using \textbf{interactive rebase} (\texttt{git rebase -i}).
  \vspace{1em}
  \begin{center}
    \hspace{-2em}
    \includegraphics[scale=0.6]{../git-figures/04-after-rebase.pdf}
  \end{center}
\end{frame}

\begin{frame}
  \frametitle{Interactive rebase}
  \begin{itemize}
    \item One of the most important Git features in the modern pull
      request-based workflow.
    \item Allows to \textbf{edit}, \textbf{reorder}, \textbf{merge}, or
      \textbf{drop} commits.
    \item \textbf{Rewrites history} -- should be only used on feature branches.
    \item \textbf{Never rewrite history of master!}
      \begin{itemize}
        \item Other developers would not be able to do\, \texttt{git pull}.
      \end{itemize}
  \end{itemize}
\end{frame}

\begin{frame}
  \frametitle{Copying commits from other branches}
  \vspace{-1em}
  It is possible to copy commits from other branches (e.g. commits implementing
  useful features from co-workers feature branches) using\, \texttt{git
  cherry-pick}.
  \vspace{1em}
  \begin{center}
    \includegraphics[scale=0.6]{../git-figures/05-recursion.pdf}
  \end{center}
\end{frame}

\begin{frame}
  \frametitle{Copying commits from other branches}
  \vspace{-1em}
  After moving 3 commits from \textit{recursion} into \textit{multiple-files}:
  \vspace{2.2em}
  \begin{center}
    \includegraphics[scale=0.6]{../git-figures/05-multiple-files.pdf}
  \end{center}
\end{frame}

\begin{frame}
  \frametitle{Copying commits from other branches}
  \vspace{-1em}
  If the commits are altered in \textit{multiple-files}, it may be needed to
  use \texttt{skip} when rebasing \textit{recursion} onto
  \textit{multiple-files}.
  \vspace{1em}
  \begin{center}
    \includegraphics[scale=0.6]{../git-figures/05-multiple-files-rebase.pdf}
  \end{center}
\end{frame}

\begin{frame}
  \frametitle{Hunting bugs in Git history}
  \vspace{-1em}
  \begin{itemize}
    \item We often discover a bug that was certainly introduced
      \textbf{somewhere in the Git history}.
      \begin{itemize}
        \item There is a revision in the past where some test works correctly.
        \item However, the test does not work now.
      \end{itemize}
    \item Git offers \texttt{git bisect} that uses \textbf{binary search} to
      localise the commit that caused the bug.
      \begin{itemize}
        \item \texttt{git bisect start} starts bisecting.
        \item \texttt{git bisect good} marks a commit that does not contain
          the bug.
        \item \texttt{git bisect bad} marks a commit contains the bug.
      \end{itemize}
    \item The process can be \textbf{automated} using a script that returns 0
      on success and a non-zero result on failure.
  \end{itemize}
\end{frame}

\end{document}
